\documentclass{article} % 文档类型为文章

\usepackage{graphicx} % 用于插入图片
\usepackage{amsmath}  % 数学公式支持

\title{LaTeX 示例} % 文档标题
\author{您的姓名}   % 作者
\date{\today}       % 日期

\begin{document}
	
	\maketitle % 创建标题
	
	\section{介绍} % 创建第一节
	这是一个简单的 LaTeX 示例文档,用于演示基本的 LaTeX 文档结构和一些常用功能。
	
	\section{数学公式} % 创建第二节
	LaTeX 是数学排版的强大工具,它能够轻松地创建漂亮的数学公式。
	
	\subsection{行内公式}
	使用美元符号可以创建行内公式,例如:$E=mc^2$ 表示能量和质量之间的关系。
	
	\subsection{居中公式}
	使用 \texttt{equation} 环境可以创建居中的公式,例如:
	\begin{equation}
		\int_{-\infty}^{\infty} e^{-x^2} dx = \sqrt{\pi}
	\end{equation}
	
	\section{列表} % 创建第三节
	使用 \texttt{itemize} 环境可以创建项目符号列表:
	\begin{itemize}
		\item 苹果
		\item 香蕉
		\item 橙子
	\end{itemize}
	
	使用 \texttt{enumerate} 环境可以创建编号列表:
	\begin{enumerate}
		\item 第一步
		\item 第二步
		\item 第三步
	\end{enumerate}
	
	\section{插入图片} % 创建第四节
	使用 \texttt{figure} 环境可以插入并引用图片:
	\begin{figure}[htbp]
		\centering
		\includegraphics[width=0.5\textwidth]{example-image} % 图片文件名为 example-image.png
		\caption{这是一个示例图片。}
		\label{fig:example}
	\end{figure}
	
	在图像 \ref{fig:example} 中,我们展示了一个示例图片。
	
\end{document}
